\section{Constitution}


\textbf{1.} The name of the society is SAGA Incorporated

\vspace{1em}

\textbf{2.} The aim of the society is to encourage the hobby of gaming, including but not limited to role-playing games, wargames, board games, and card games, and to aid its members in the pursuit of these hobbies. 

\vspace{1em}

\textbf{3.1} A person becomes a member by paying the annual subscription fee to an officer of the society, providing the required information, and receiving a receipt.

\textbf{3.2} There are two classes of members:

\begin{itemize}
  \item Student members, who must be current members of the University of Canterbury Students Association; 
  \item Non-student members, who must be at least 16 years of age. 
\end{itemize}

\textbf{3.3} The rights and privileges of membership commence on payment of the annual subscription fee, and end on the day before ``Clubs Day'' of the following year.

\textbf{3.4} The annual subscription fees for each class of member will be decided by the active executive at the beginning of the year.

\vspace{1em}

\textbf{4.1} Membership may be resigned by giving written notice to the secretary. 

\textbf{4.2} Non-members may be permitted to attend the activities of the society. The executive committee is to decide on the regulations that will be enforced on such occasions.

\textbf{4.3} Membership may be suspended or revoked by a majority vote at a general meeting.

\textbf{4.4} Membership ceases on the death of the member.

\vspace{1em}

\textbf{5.1} The Secretary must make available a copy of this constitution to any member upon request.

\textbf{5.2} This constitution can only be altered at a general meeting of the society.

\textbf{5.3} A motion to alter the constitution requires a three-fourths majority vote in its favour to be carried

\textbf{5.4} No addition to or alteration or recession of the Constitution shall be approved if it affects the non-profit aims (Section 2), personal benefit clause (Section 11), or the winding up clause (Section 12).

\textbf{5.5} Notice of proposed constitutional changes must be advertised to the society and delivered to the President at least 7 days prior to the general meeting at which they will be voted upon.

\vspace{1em}

\textbf{6.1} The Annual General Meeting of the society shall be held during the third term, unless circumstances prevent this, in which case it shall be held as near to this date as is reasonable.

\textbf{6.2} Notice of an Annual General Meeting must be advertised to the society at least fourteen days prior to the meeting.

\textbf{6.3} The quorum for an Annual General Meeting is 15 members or 25\% of the society membership, whichever is less.

\textbf{6.4} Members may submit items for the agenda at least 7 days before the AGM. 

\vspace{1em}

\textbf{7.1} A Special General Meeting may be called by the executive committee, and must be called when requested by 15 members or 25\% of the society membership, whichever is less.

\textbf{7.2} A Special General Meeting may deal with any business normally dealt with at an Annual General Meeting, including election of officers to the executive committee.

\textbf{7.3} A Special General Meeting must be held no earlier than seven days and no later than fourteen days after the decision to hold it has been made. 

\textbf{7.4} Notice of a Special General Meeting must be advertised to the society at least seven days prior to the meeting.

\textbf{7.5} The quorum of a Special General Meeting is 15 members or 25\% of the society membership, whichever is less.

\textbf{7.6} Members of the executive may be removed from their position by a majority vote in a General Meeting.

\vspace{1em}

\textbf{8.1} General meetings will be chaired by the president if available, or by another officer of the society.

\textbf{8.2} The chairperson has a casting vote should equal voting occur at a meeting.

\textbf{8.3} All voting at meetings of the society is by a show of hands unless a secret ballot is called for, in which case such a request is always successful.

\vspace{1em}

\textbf{9.1} The executive committee of the society consists of five officers: the President, Secretary, Treasurer, Quartermaster, and Promotions officer.

\textbf{9.2} Executive officers are elected at the Annual General Meeting of the society, and take office on the 1st of January the following year, and hold office til their resignation or the 31st December that same year.

\textbf{9.3} The executive committee has the authority to create bylaws within the bounds of this constitution.

\textbf{9.4} Only members of the Society are eligible for election to and to hold office in the executive committee.

\textbf{9.5} An executive officer must act in accordance with policy decided at a general meeting of the society.

\textbf{9.6} The President is responsible for organising the activities of the society and acts as chairman at all meetings of the society. Should they be absent any committee member may take the chair if directed by members present.

\textbf{9.7} The Secretary is responsible for all administrative duties of the society, notably the taking of minutes, issues of notices and conducting of club correspondence.

\textbf{9.8} The Treasurer is responsible for the finances and assets of the society, notably the collection of annual subscription fees and all matters dealing with society purchases.

\textbf{9.9} The Promotions officer is responsible for promoting the club. This responsiblity covers the publication of any materials for the club, advertising required for the club, and maintenance and upkeep of the club's social resources.


\textbf{9.10} No officer shall be given a responsibility at a meeting where they are not present, unless they have already expressed to the committee a wish to accept that responsibility.

\textbf{9.11} The Quartermaster is responsible for maintenance and inventory of the items normally stored in the Society lockers.

\textbf{9.12} An officer may resign their position by giving written notice at a committee meeting. 

\textbf{9.13} If an officer resigns a general meeting must be held within 4 weeks to elect a new officer.

\vspace{1em}

\textbf{10.1} The common seal of the society will be held by the secretary.

\textbf{10.2} The common seal may only be used after a successful motion at a committee meeting.

\vspace{1em}

\textbf{11.1} The funds of the society are applied only towards the promotion of the objects of the society. No portion of the funds is paid directly or indirectly to any member of the society, except as reimbursement for expenditure approved by the committee, upon presentation of proof of that expenditure.

\textbf{11.2} The funds of the society will be handled by the treasurer, as directed by the committee. The bulk of society funds should be held in an appropriate financial institution. 

\textbf{11.3} The accounts of the society must have two signatories from the committee for any withdrawals and cheques, one of whom must be the Treasurer or the President.

\vspace{1em}

\textbf{12.1} The society can be wound up by a simple majority at a general meeting. The resolution must be confirmed by a subsequent general meeting to be held not earlier than 30 days after the date of the resolution to be confirmed.

\textbf{12.2} If upon the winding up or dissolution of the society there remains after the satisfaction of all its debts and liablities any property whatsoever the same shall not be paid or distributed among the members of the society but shall be given or transferred to, or placed in trust for, some other organization or body having objects similar to the objects of the society, or to some other charitable organization or purpose, within New Zealand.

\vspace{1em}

\textbf{13.1} Meetings of the executive committee can be called by any executive officer. All executive officers must be notified of the meeting at least one week before the meeting. A meeting may be held earlier if all committee members agree.

\textbf{13.2} The quorum for a meeting of the executive committee is three executive officers. Any motion made at such a meeting requires at least three votes to pass.
