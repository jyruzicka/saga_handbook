\section{Incoporated status}
\label{sec:incorporation}

We didn't just get the ``Inc.'' at the end of our name by wishing: SAGA Inc. is officially an incorporated society. This is handy as it means that if the club goes massively into debt, the committee members are not personally liable. There are probably some other law-related advantages too, but that's beyond my expertise.

All incorporated society paperwork can be done through the Societies website\footnote{http://www.socities.govt.nz}. While the interface is a little antiquated, it still works. As an incorporated society, we have a couple of obligations (at the time of writing):

\textbf{To keep a constitution} (technically called rules). This means that if you change the constitution you must notify the Companies Office. Instructions for what you need to do will be available on the Companies Office website\footnote{Currently: http://www.societies.govt.nz/cms/incorporated-societies/running-a-society/how-do-you-change-the-rules-of-a-society}.

\textbf{To submit financial statements}. You must submit financial statements to the Companies office each year\footnote{Details: http://www.societies.govt.nz/cms/incorporated-societies/financial-statements}. This is the combined job of the secretary and treasurer (at least, with the distribution of roles laid out in this manual it is). By the end of the year you should have submitted one of these: it's the treasurer's job to make the report and submit it, and the secretary's job to follow up that the treasurer has done so.

You can also do the following through the Companies website:

\textbf{Change details}. This includes the contact details for committee members. While you have no contractual obligation to change these details, it's nice to keep the Companies office up to date on who to contact for club matters.