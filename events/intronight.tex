\section{Intro night}

The first two nights of SAGA are usually oriented towards getting new members into games, or at least integrated with the club. It's a great chance for GMs to advertise upcoming campaigns as well.

\subsection{Preparation}

Make sure you have a few GMs running games. Two is OK, three is better, four is great. You want to make sure that new players get into a game so they can see what this is all about. Get in touch with the good GMs at SAGA and see if they'll run something. Again: if you talk to people individually, you're more likely to get a response than if you just send out a generic message to the club at large.

Make sure all the committee are turning up to intro night too. This should be a no-brainer, but check anyway.

\subsection{On the night}

It's very easy for new members to be intimidated by the club. You should do everything you can to make sure new members (or potential new members) feel like they're part of the fold as soon as possible.

If you're a committee member, this means you should keep an eye out for new people, or people who don't look like they belong. Go up to them, introduce yourself, ask what sort of games they'd like (or if they don't know, maybe what sort of games they \textit{think} they'd like). Introduce them to a few other members if you can (if they like board gaming, and you know someone who organises a lot of board gaming, introduce them! If they want to roleplay, and you know a GM who's looking for players, introduce them!). Ideally, you should be able to get them into a circle of people, then you'll be able to move onto the next lost soul.

The committee should introduce themselves both nights. Say who you are, what you do on the club, what sort of games you play, all that. Also mention to the new members what it is we do, how we operate, re-iterate why they want to be members, and all that.

It's traditional for GMs to advertise their campaigns for the year at SAGA. While you're listening to the campaign blurbs, it's sometimes good to pick out the ones that are overly technical and get the GM to explain what they mean.

Following campaign blurbs, you will need some way of getting one-off games going. The standard way of doing this is to have each GM say what their one-off game is going to be, and then letting people converge on the potential GMs like rabid wolves: unfortunately, this means the shy newbies often end up at the rear of the mob.

One way around this is to have a ``hands-up'' system of interest. Once the games have been advertised, go back through them and ask who'd be interested. By a show of hands you can work out whether there's enough interest, and who's interested in playing what. You can also guarantee newbies get a spot in the game they want by giving them priority over veteran members.