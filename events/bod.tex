\section{Running Buckets of Dice}
\label{sec:buckets_of_dice}

Buckets of Dice is the club's annual convention and the biggest thing you'll need to do for the year. It usually takes place on Queen's Birthday weekend, but there's nothing stopping you changing it if another date suits you better.

I will talk about some of the ``big-picture'' things you should consider, and then provide a time-line leading up to the convention with tasks that should be done.

\subsection{Things to consider}

\subsubsection{Format}

Decide early on the format of the convention. This includes:

\begin{itemize}
  \item When the big events (Live Action, Grand Strategy) are
  \item How long each roleplaying session is
  \item What time it starts and finishes
  \item Whether or not events run onto the Monday
\end{itemize}

Roleplaying sessions have tended to be between three and four hours long. Three hours means GMs need to be on the ball, starting on time and getting right to the action, but also means it's possible to fit three sessions into a day and be done by 7pm--8pm. Four-hour sessions give GMs lots of time for their adventure but also mean that you're limited to two sessions if you want an event in the evening.

\subsubsection{Big events}

Buckets is traditionally headed by a live action RPG and a grand strategy (massive wargame). It's important to get volunteers to write and run these events as soon as possible. Whoever did it last year is often a good person to bug about either doing it this year, or finding someone else to do it this year. It's \textit{strongly advised} that whoever is running these events be not on the committee as well: organising the convention \textit{and} running one of the big events is a considerable load.

Other things you may wish to consider about big events:

\begin{itemize}
  \item Which nights will they run?
  \item How will sign-up work? Will people sign up for these events when they sign up for the convention?
  \item What budget does each of these events get?
  \item Will you award prizes based on these events? If so, you may want the event to happen early in the weeekend.
\end{itemize}

\subsubsection{Billeting and other clubs}

The more people come from out of town for Buckets, the better. More out-of-towners means more varied games, more interaction between centres and generally a healthier national roleplaying environment.

We can encourage out-of-towners to come to Buckets of Dice by making it easier for them to attend. One of the most cost-effective ways of doing this is by organising billeting: that is, getting SAGA members to host out-of-town attendees so they don't need to worry about accommodation or getting to the convention.

You will need to organise billeting before sign-up, since people will want to know if they have accommodation before saying they'll come along. In general it's good to get billeting space for four or five people, perhaps more if you know there's a big group coming up.

If you do get billeting, it's a good idea to contact other clubs with this news. If you don't know how to get in contact with other clubs, ask around the older members of SAGA  they will probably know someone who knows someone.

As of the time of writing, the other major roleplaying clubs around New Zealand are:

\begin{itemize}
  \item OURS (Otago University Roleplaying Society): run through Facebook
  \item Victoria University Gaming Society: Facebook
  \item WARGS (Wellington Area Roleplaying Game Society): most activity through http://nzrag.com
  \item AMERICA (Auckland something something something): http://www.theamericaclub.net.nz/
\end{itemize}

\subsection{Timeline}

\subsubsection{Twelve months out}

\textbf{Decide on a time and venue.} As soon as possible, you should arrange a date and venue. Since the 2011 earthquake we've been running from Otakaro Building the in College of Education, which is effectively free and lets us keep costs to a minimum. As of 2013 Otakaro has been closed for structural repair, but the Wheki building in the College of Education is still available.

The sooner you have a date and venue, the sooner you can advertise this.

\textbf{Form a subcommittee.} The old subcommittee should be doing this, so it's not your thing to do, but I'm putting this here anyway. The sooner the subcommittee is formed, the better. People can always be added later on (you don't always know if you'll be in town in a  year), but having a subcommittee head at this stage is really handy. I suggest you put people in charge of the following:

\begin{itemize}
  \item Liasing with/securing sponsors
  \item Liasing with LARP organisers
  \item Liasing with Grand Strat organisers
  \item Organising venue
  \item Organising GMs and/or timetable
  \item Promoting the event
  \item Printing
  \item Finance
  \item Applying for a grant
  \item Purchasing food \etc
  \item Organising signup
  \item Coordinating out-of-town attendees
\end{itemize}

These don't all need to be separate people: the LARP and Grand Strategy liason could be the same person, or you could have one person who's on finance and venue, or what-have-you. However, it's good to have someone be explicitly aware that \textit{this} is the thing they're responsible for.

\textbf{Ask for nominations for grand strategy and live action.} People are pretty pumped from the last LARP and grand strat at this stage. Advertise to the club that you're looking for people to run next year's events. You'll need to get a blurb from them so you know what the game's about.

\subsubsection{Eleven months out}

\textbf{Re-announce LARP and grand strategy submissions.} Remind people that submissions will be closing soon.

\subsubsection{Ten months out}

\textbf{Close submission for LARP and grand strategy.} Once you have your submissions, the subcommittee gets to decide which they're running. You can then announce this to the club and the lucky volunteers so they have ten months to write the game.

\subsubsection{Summer holidays}

\textbf{Design a promotional flyer.} This might entail getting a logo (can anyone on the committee draw? Do you know anyone who can?), making a little A5 thing with a title and an image, putting promotional blurbs about the LARP/grand strat down, letting people know the date and venue. You'll need these to take up to KapCon or hand out at clubs day.

\textbf{Talk to people about sponsorship and discounts.} This is a good job for someone who wants to be involved on the committee but doesn't want too much to do. Start talking to everyone you can about sponsoring Buckets, offering prizes or discounts. You may wish to talk to:

\begin{itemize}
  \item Local gaming businesses (\eg Wizard's Retreat, Comics Compulsion, The Game Depository)
  \item National web gaming businesses (\eg Seriously Board)
  \item Local book-sellers or other peripherally-related business (\eg Scorpio Books)
\end{itemize}

Discounts are good, prizes are better. Book vouchers or vouchers for games are fine.

\subsubsection{January}

\textbf{Take posters to KapCon.} Someone you know is going to KapCon, right? If so, give them some posters and/or flyers to put up when they're up there. If no one's going up, maybe you could post some up.

\subsubsection{February}

\textbf{Assemble a budget.} A sample budget is included in this document.

\textbf{Apply for a grant.} Once you have a budget, you can apply for a grant from the UCSA. This helps considerably with costs.

\textbf{Ramp up advertising.} Advertise the event at clubs day. Advertise it in OOC. Advertise it on the website and Facebook. Make sure people are aware of it.

\subsubsection{March}

\textbf{Continue promoting the event.} Mention it's happening at weekly meetings, mention it on Facebook again, \etc~ \etc.

\textbf{Call for GMs.} You always need GMs. This is a good time to personally approach good/experienced GMs and ask them what they'll be running. Once again: talking to people one-on-one is far more effective than throwing out a blanket email and hoping people will respond.

Generally, we have people tell us if they're running a game when they sign up. You may want to have people tell you what games they're running before signup, so you have a list for signup itself.

\textbf{Buy prizes from Amazon and overseas stockists.} It's good to have prizes at events like these, and it's even better to have prizes that people actually want. If you need to order stuff from overseas, you should be doing it now so it arrives in time.

\subsubsection{April}

\textbf{Registrations open.} Ideally you want registrations to open at this stage, so people can get signed up. If you get a bunch more people than you have games, you then have time to run around and recruit more GMs. This also means that you can get player information to the people running the LARP and Grand Strat relatively early on.

\textbf{Continue promotions.} You can now advertise that sign-up is open. You might also want to advertise to a wider audience, \eg through chalking on campus, adverts in Canta, or posters on bollards.

\subsubsection{Two weeks out}

\textbf{Registrations close.} People can still sign up on the day but early registrations (including those for the LARP and Grand Strat) close at this point.

\textbf{Buy local prizes.} If you're buying prizes from within New Zealand or Christchurch, you can buy these now if you haven't already.

\textbf{Buy props and materials for the LARP and Grand Strat.} Check with the event-runners to see what they need. If you buy it now you won't be running around trying to get stuff a couple of days before the event. Note how close to budget you're running on these.

\textbf{Start printing.} You're going to need to print a lot of material, so you'd best start now. You'll need flyers and the like for regular attendees as well as instructions, tokens, character sheets, maps, \etc~ for the LARP and Grand Strat. Here are some things you might want to print out for the convention itself:

\begin{itemize}
  \item Signage ("Convention entrance this way")
  \item Brochures (including sponsors, roleplaying sessions, LARP and grand strat blurbs, map of the venue, local places to get food, timetable, emergency contact numbers)
  \item Checklists for people on desk
  \item Roleplaying session signup forms
  \item Best GM/best player nomination forms
  \item Emergency contact list for people on desk
  \item If you're providing meals, meal signup forms
  \item Signup sheets for people who want to sign up to SAGA on the day
  \item Out of Character (BoD issue)
  \item ``On-the-day'' sign-up sheet
  \item Additional expenses form (in case someone has to go out and buy sellotape and wants to be reimbursed)
  \item A voting tally form
  \item Name badges for pre-registrations
  \item A record of people who've pre-registered and paid
\end{itemize}

\textbf{Organise a desk schedule.} Who'll be on desk when? Hopefully you can limit everyone to being on desk for one session. Don't be afraid to ask other SAGA members to help: the worst they can do is say no. Again, asking people one-on-one is most effective. You could always budget in some lollies for people on desk.

\textbf{Continue promoting.} Remind people that they can always sign up on the day.

\textbf{One week out}

\textbf{Buy food.} Are you planning on selling food at the convention? You will want to buy it around now.

\textbf{Finish off any additional printing.} Double-check with LARP and Grand Strat organisers in case they've thought of anything else that needs to get printed off.

\textbf{Print up a desk schedule.} Make sure everyone who's on desk knows when they're on, and also when everyone else is on.

\textbf{Check up with the venue.} Make sure you have keys (if you need keys to get in). Make sure they know how late you plan on being. Check to see if you need to be out of there by any particular time. Make sure they know how to contact you (and you know how to get in touch with them) if anything goes wrong.

\textbf{Assemble registration packs.} If you have bags, you can prepare packs for those people who've preregistered.

\textbf{Final promotion.} You didn't think you could stop doing this yet, did you?

\subsubsection{The day before}

At this point it is \textit{incredibly} handy to have someone with spare time and a car. If someone can drive around on the Friday and organise everything you'll have a valuable buffer. If you don't have this luxury, you may want to move some of these items to the ``Week before'' list.

\textbf{Do any miscellaneous last-minute tasks.} Have people go through their areas of responsibility. Is there anything they've neglected to do? Anything they've just thought of? Now's the time to get it done.

\textbf{Organise paperwork into folders.} Buy a pack of manilla folders, write or label each, and sort the paperwork appropriately. You'll appreciate it when Buckets is in full swing and you need to get to the new membership list.

\textbf{Organise incidental equipment.} Are you running SI-FI\footnote{SI-FI is the SAGA Inc. Fruit Initiative, in which attendees donate money and fruit is provided to all convention-goers.}? Do you have anything else going on that the committee is organising? Make sure you have what you need to keep it going. For SI-FI we needed bowls, knives and a chopping board. It's a good idea to run through the event in your mind to see what you'll need. Regardless, we've found the following are handy to have: scissors, Sellotape, blue-tack, rubber bands, permanent markers.

\textbf{Secure a float.} You'll need change for people who want to pay for biscuits with \$5 notes and the like, so make sure you have change on hand at the desk. \$50 worth of money, in \$5 notes and coins, should be sufficient.

\subsubsection{The day of}

\textbf{Ensure everyone is there on time.} If you're helping run the event you probably need to be there at least half an hour before signup opens, maybe more depending.

\textbf{Put up posters and mark rooms.} Make sure people can work out where the convention is. Label the rooms you're using so people know how to get to them. People who're on desk should know where the rooms are anyway, so they can direct lost children and confused con-goers.

\textbf{Set up the desk.} For the initial rush you'll probably want a couple of tables going with two or three people on desk processing signups. Here is what you're in charge of when you're on desk:

\begin{itemize}
  \item Keeping track of who has paid for what (food, signup, drink)
  \item Last-minute LARP and Grant Strat prep (\eg cutting up nametags)
  \item Preparing for the next session (preparing sign-up sheets, getting rid of old paperwork, \etc)
  \item Taking care of the float
  \item Collecting receipts if people need to buy things for the club
  \item Answering questions
  \item Being in charge
  \item Knowing what's going on
\end{itemize}

\textbf{Coordinate food orders.} It's generally advisable to close food orders around the end of lunch time. This way you can phone in an order early in the afternoon and go pick it up just before dinner.

\subsubsection{After the convention}

\textbf{Breathe.}

\textbf{Make sure you've cleaned everything up.} Especially around the sites of the LARP, Grand Strat. Make sure if you ordered pizza in, that all the boxes are cleared up. You may want to coordinate with security if you have a lot of pizza boxes.

\textbf{Collect feedback.} Let people fill in surveys and find out what went wrong. This way you can find out what you need to do next year to make your convention better.