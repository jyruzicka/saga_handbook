\section{Running clubs day}
\label{sec:clubs_day}

Clubs day technically happens twice a year, at the start of each semester. In practice, the number of people you recruit at the second clubs day is tiny: if you have people spare around this time, or you really need members (or you have something awesome to sell) it might be worthwhile, but historically we haven't found it worth the time required to organise it.

The UCSA \textit{should} let you know when clubs day is approaching, but it's generally a good rule to assume that the UCSA will forget to notify you about things. Around January you could consider emailing the UCSA about when it's scheduled and anything you need to do as a club.

\subsection{Documents}

Here are the things you'll need to have on hand at clubs day:

\begin{itemize}
  \item \textbf{Propaganda:} This includes (hopefully) a first edition of \textit{Out of Character} for the year but might also include promotional material for sponsors or upcoming events. Generally around $\sim$100 copies, to give out to anyone who displays interest.
  \item \textbf{Membership cards:} For members. A set of 100 should do you throughout the year.
  \item \textbf{Signup form:} Make sure you collect relevant details for the club and also the UCSA. It's often good to collect (at the least):
    \begin{itemize}
      \item Name
      \item Email
      \item Student ID
    \end{itemize}
    \item \textbf{Interest form:} for people who are interested but don't have the money/don't want to sign up quite yet. You only really need email addresses here, but it means you can follow up with a bunch more people when SAGA events come up.
    \item \textbf{Petty cash:} change for people who want to pay for their membership with twenties. A bunch of \$5 notes is really handy to have.
    \item \textbf{Receipt book:} for proof of payment etc. Everyone who pays you money gets a receipt, and your treasurer gets confirmation that people have paid money. Handy for paper trails.
\end{itemize}

\subsection{Dressing the table}

It's a good idea to have a bunch of shiny board games and roleplaying games on the table to draw people in. Don't worry about which games are \textit{technically} good at this juncture, have the things that look the prettiest with the biggest models or whatever. You can also use SAGA's banner to make the table look pretty.

\subsection{Scheduling and rosters}

Before the event you want to round up people to do time on the desk. Try to get the more outgoing members of the club to volunteer: if everyone on desk is staring at their feet all day, you won't get many new signups.

People shouldn't need to be on desk for more than a morning or afternoon. It's a good idea to have at least two people on desk at any time. You'll get more people if you ask them personally (but you still want to put out a general call to the club), so identify the people you'd most want to sell your hobby to the masses and make sure you personally ask them (\textit{via.} email, phone call, face-to-face) if they can spare the time to sit on the desk.

\subsection{Selling the club}

You're allowed to be a person. Get a name, find out what they're studying, connect, do all that.  There's still a vast array of people who don't know what \textit{Dungeons \& Dragons} is, so you'll probably want to work out what their level of exposure to this sort of game is. If they've never played anything before, maybe that's when you show them how something like \textit{Fiasco} works, rather than starting to talk about how the new \textit{Dresden Files RPG} encourages character-driven play by use of aspects.

If people want to board game, you can always highlight how we regularly buy new games and have a collection in the lockers. If they want to roleplay, outline some of the more awesome campaigns that have been run at our club, or \textit{Buckets of Dice}. Don't forget discounts from our sponsors either.

Basically, don't be the stereotypical \textit{D\&D} nerd, find out what they want from the club, and show them how we can give it to them.

\subsection{Follow-up}

Once you have a bunch of emails, use them. Email people a few days after clubs day and let them know what's happening. The club currently has a subscription to \textit{Mailchimp}, a mass-mailing service that allows you to send pretty damn professional-looking emails out to a vast number of people. With any luck, a decent-looking email will get you noticed amongst the however-many other club emails, and get new people along to the introduction nights.

Don't be afraid to add your ``vague interest'' list to the Mailchimp database either. Just make sure you don't spam people. One email at the start of the year and maybe one in the lead-up to Buckets of Dice is fine. You should make sure that non-members know they're welcome at the club's introduction nights and any other early-year one-off events.

For more information on using Mailchimp, see the Assets section of this manual.