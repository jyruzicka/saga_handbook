\section{Promotions Officer}
\label{sec:promotions}

The promotions officer deals with promotion of the club. This includes maintenance of the club's various assets for talking with its members (website, mailing list, facebook page), production of any harcopy promotional material, and regular editing and publication of Out of Character, the club's semi-regular publication.

\subsection{Key Tasks}

\begin{itemize}
  \item Publish Out of Character
  \item Maintain promotional assets
  \item Promote SAGA events
  \item Promote SAGA as a club
\end{itemize}


\subsection{In Depth}

\subsubsection{Publish Out of Character}

Out of Character is SAGA's semi-regular publication, usually containing a number of articles by its members about what's going on at the club. As promotions officer, you get to edit these articles together into one cohesive whole and get the thing published to the club.

Getting people to contribute to OOC can sometimes be a challenge. Do not be afraid to bug people who've promised you articles. Talk to people in regular games and find out what they like about their current game, then hopefully you can get them to write about stuff.

Publication is generally done as an A5 booklet, printed double-sided. At the time of publication, SAGA has an account with the UCSA for cheap printing: talk to your previous Promotions Officer to obtain SAGA's printer code, or talk to the UCSA if you've lost it.

Your first OOC for the year should be ready for Clubs Day, and can include information such as upcoming campaigns, profiles on the committee for this year, a timetable of upcoming events, a list of sponsors, etc. It's a good idea to print somewhere between 80--100 copies of Issue One, as you'll want to hand it to people at Clubs Day. For further issues, you only tend to need around 40 copies.

It's often a good idea to leave a stack of copies of OOC at places like Comics Compulsion at Northlands. This way, non-club-members can get additional exposure to the club through issues. Make sure non-members know where to find details on the club!

\subsubsection{Maintain promotional assets}

This includes the Facebook page, website and mailing list. You want to keep people informed of what's going on: when you have upcoming events, let people know in as many ways as possible. Make sure the website has up-to-date information on the committee and meeting times/places, etc. Become familiar with your tools, and how to use them. You should be able to get usernames/passwords from the previous Promotions Officer --- if not, most services have password recovery services or the like.

If you feel lost using this software, organise a time with last year's Promotions Officer to sit down with them and go through everything. You won't regret it.

\subsubsection{Promote SAGA events}

Related to the previous section: when SAGA is holding an event, you should make sure that everyone in the club knows about it. Facebook is a good start, but you can also email members through the mailing list and post about it on the website. You could also get advertising at some of SAGA's sponsors: Comics Compulsion is usually fine with use putting up adverts in their shop, for example.

\subsubsection{Promote SAGA as a club}

If you feel adventurous, you can always do some advertising for the club past clubs day. It's tricky to advertise to roleplayers, because once you have a rulebook and a group you don't need much else - there's no regular requirement to buy new stuff or meet other roleplayers. Nevertheless, by advertising in the right places --- for example, targetted advertising on Facebook or by putting up fliers around University or other venues --- you may be able to reach more people than you usually would. Having an advertisement in Comics Compulsion is often good, since this is Christchurch's main local gaming store.