\section{Treasurer}
\label{sec:treasurer}

The Treasurer's role is to make sure that the club remains financially in the black, while also providing money for asset purchases and Buckets of Dice. As SAGA is a not-for-profit society, the goal is not to make money: any profits will ideally be put back into the club in the manner of board or role-playing games, additional events, or the like.

\textbf{Note:} A number of the documents contained at the end of this manual will help the treasurer keep track of expenses throughout the year.

\subsection{Key Tasks}

\begin{itemize}
  \item Keep track of petty cash and banking
  \item Create a budget for the year's expenditure
  \item Submit grant applications
  \item Authorise and record expenditure throughout the year
  \item Organise finances for Buckets of Dice
  \item Oversee changing signatories at the end of the year
\end{itemize}

\subsection{In Depth}

\subsubsection{Keep track of petty cash and banking}

SAGA maintains a supply of petty cash for small expenses. It is the treasurer's job to ensure that there is a good amount of cash for this --- excess can be deposited into the club's bank account.

The club also has two bank accounts: an everyday account and a high-interest online call account. Both of these accounts require at least two signatures from committee members for any withdrawals \textit{etc.}. Both accounts are with Kiwibank, and you can check balances and make/confirm payments online.

\subsubsection{Create a budget for the year's expenditure}

The largest expenditure for the club will usually be hiring a venue for Buckets of Dice, and the club should have an idea of how much this will cost around the start of the academic year. Other costs will include barbecue and boardgame afternoons and any other events organised by the club.

The main source of income is, of course, membership fees, although a good amount of money will also come from entry to Buckets of Dice. Membership fees are used to pay for barbecue and boardgame afternoons and club purchases, while Buckets admissions are used to pay for Buckets itself.

It is up to the treasurer to make sure that membership and admission fees are low enough to make the club attractive to new members, while being high enough to prevent any financial losses to the club.

\subsubsection{Submit grant applications}

Through a grant, we can receive money from the UCSA to help with expenses. We need to submit an application to the UCSA in order to get money from them, and we need to do it \textit{before} we start spending the money. It's a very good idea to get a grant for Buckets of Dice to help us with our expenses. You should apply for a grant as soon as you have a budget for BoD. Grant paperwork (including application forms and UCSA policy documents) can be found on the UCSA website\footnote{http://ucsa.org.nz/clubs/documents/}.

\subsubsection{Authorise and record expenditure throughout the year}

As treasurer it's also your job to make sure that the club spends its money wisely. You should be consulted on all financial decisions the club makes.

It is important to record any club purchases and expenses throughout the year, both to keep our accounts up to date and to present to the rest of the committee. The treasurer is expected to make a report at the AGM, and so it's important to know where the money's going, whether the club made a loss or a profit, and give an overall idea of the club's health.

A good way of keeping track of expenditure throughout the year is to have an Excel spreadsheet of income and expenditure. By recording date, details, expense, income, and total profit/loss for \textit{everything} the club does (including, for example, printing, membership, events, website expenses, \etc) you can make a much better financial report at the end of the year --- plus when it comes to September you'll still have a good record of what happened in March. This isn't hard! A sample cashflow record is available on p\pageref{sec:cashflow}.

\subsubsection{Organise finances for Buckets of Dice}

Buckets of Dice will likely cost somewhere between \$500 and \$1000, with money required to book the venue, reimburse contributors, buy food and drinks, and provide prizes. The treasurer should prepare an estimate of the cost of the event, and set ticket prices accordingly.

Additionally, the treasurer should submit a request to the UCSA for a grant as soon as possible. The UCSA will typically grant the club half of the cost of Buckets.

When you're preparing for Buckets of Dice, keep track of individual expenses and income. A cashflow spreadsheet might be handy here. It may feel a bit pedantic to keep track of how much money (to the cent!) you spent on chips and drinks for Buckets, but you'll feel a lot better if you suddenly find you have \$200 less than you thought you did: you can quickly check to see if you've overspent in an area, or if you're genuinely missing cash.

At the end of Buckets, you should also prepare a report of income and expenses so the committee knows how much it cost in general.

Sample documents for budgeting and reporting actuals for Buckets of Dice are available in the Documents chapter (p \pageref{cha:docs}).

\subsubsection{Oversee changing signatories at the end of the year}

It is important to keep the list of signatories to the club's bank accounts up to date, and it is the treasurer's job to ensure that when the committee changes, so do the signatories for the accounts.

Changing signatories will involve collecting a form from the Clubs Development Officer at the UCSA and getting every committee member to fill it out and sign it. This can then be taken to the bank and the requisite changes to the accounts made.