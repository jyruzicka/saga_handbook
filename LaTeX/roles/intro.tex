\section{Introduction}

The SAGA committee is currently made up of five members, each of which has responsibility for one section of the club. These are:

\begin{itemize}
  \item \textbf{The President}, who runs the club and coordinates the committee.
  \item \textbf{The Secretary}, who deals with other bodies as SAGA's representative, reminds the president about everything they should be doing, and generally acts as the President's right hand.
  \item \textbf{The Treasurer}, who tracks the club's finances and authorises purchases/expenditure.
  \item \textbf{The Promotions Officer}, who is in charge of publicising the club's activities to its members and the general public.
  \item \textbf{The Quartermaster}, who keeps track of SAGA's assets (board and roleplaying games) and advises the committee on new purchases.
\end{itemize}

The rest of this chapter is a more in-depth role of each position, outlining the typical responsibilities each committee member should take in the club. It should be noted that just because we say a committee member is responsible for something, doesn't necessarily mean that they're the person best suited to the task. You should delegate and outsource these tasks as you see fit, but remember to make sure that it gets done at some point.

At the time of writing these lists are still in flux. If you find yourself having to do something that isn't on your list, or if you think one of the items on this list would fare better as the responsibility of a different role, get in touch with me and I will add it to the next iteration of the SAGA manual.

Additionally, it's generally a bad idea to assume that your responsibility stops when your role does. A good committee communicates and is aware of what the rest of it is up to. You all have a responsibility to know (at least peripherally) what's going on, in case your treasurer gets run over\footnote{This happened once} or your president suddenly ends up overseas for several months\footnote{This happened once too} or your quartermaster suddenly decides to resign and move to Napier\footnote{See previous footnotes} or every other committee member gets tragically gunned down in a gangland murder\footnote{OK, this one hasn't happened...yet.} \textit{you know what's going on}. You also have a responsibility to tell everyone else what you're doing, so that if this happens to you, they can cover for you. For you, this might mean regular meetings with well-appointed minutes and a clear paper trail, or it could mean daily updates to a Facebook group or mailing list. The important thing is that you should keep in contact with your fellow committee members.

\subsection{Giving committee tasks to non-committee members}

Do it.

People become SAGA committee members because they want to contribute to the club, and they're confident in a role of responsibility. We can make people want to contribute by being an awesome club which does cool things for its members, and we can help those members feel confident in roles of responsibility by giving them little tasks to do --- not big things like actually being on committee --- but little things they can do to help out. If someone in the club actively wants to help, that's your chance to give them little things to do. Have them come to meetings with the committee, or help run things and see how everything works behind the scenes.

It's also good practice to let these people ``apprentice'' or ``understudy'' roles in the committee. If an eager but inexperienced member gets to see what committee members actually do day-to-day, and gets familiar with those roles, they'll be a lot more confident running for committee at the next AGM.