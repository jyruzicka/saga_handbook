\section{Petitioning to change University policy}

SAGA Inc. is a University-affiliated club. This is good news for us: we get a lot of new members, free bookings, and the ability to apply for grants through the University of Canterbury, and these are all Very Good Things.

However, sometimes University policy can come into conflict with SAGA's values or activities. While the University can \textit{feel} large, it's still very possible for clubs (especially clubs who act in a concerted and adult manner) to influence University and UCSA policy.

In this section I will highlight the steps you need to go through if you want to help change University policy towards clubs.

\subsection{Determine whether change is necessary and realistic}

The first step is to decide, as a committee, whether you believe this policy can and should be changed. You have the best change of changing policy if you can show that it is unfair or disadvantages other clubs than just yours.

It's also worth noting that protesting policy every three months is a good way to get the UCSA pissed off at you. Dealing with this sort of thing takes work on their end too.

\subsection{Reach out to other clubs}

Do you know members of other clubs that are affected? Having more than one club alongside you is a great way to show the UCSA and the University that it's not just one greedy club trying to benefit from policy change.

If you don't personally know anyone in other clubs, you can get club contact information from the UCSA clubs page\footnote{http://ucsa.org.nz/clubs/}.

\subsection{Draft a position}

Meet up and work out:

\begin{itemize}
  \item \textbf{why} you believe current policy is unfair (including any concrete examples you happen to have)
  \item \textbf{how} this disadvantages clubs in general, or a specific subset of clubs
  \item \textbf{what} might be done to fix this policy
\end{itemize}

\subsection{Talk to the UCSA}

The UCSA can be your greatest ally in this case. Find out who your UCSA representative is\footnote{At the time of writing, this can be found at: http://ucsa.org.nz/clubs/documents/} and send them an email. Remember, however, that it generally isn't the UCSA's fault that the policy is how it is. Your goal should be assertive without being aggressive or confrontational. Sign your email with the names of the clubs who support you.

If you're lucky, this should be as far as you need to go. The UCSA may talk to you about alternatives, and should then get in touch with the relevant people at a higher level.

If the UCSA decides that policy is best the way it is, this might be a good place to re-evaluate your position. Despite what you may feel in your gut, the UCSA won't decline your request just because they hate you, or they think it'll be actual work to pass the policy change. You should be able to get a reason out of then, and this may make you realise that while the policy isn't ideal for your club, it's better than any alternative.

If you still feel like the policy should change (despite advice from the UCSA), you can proceed further.

\subsection{Escalate to the policy's contact person}

In order to escalate further, you have to find the policy's contact person. Every policy is available on the UC's policy library\footnote{http://www.canterbury.ac.nz/ucpolicy/}, and each policy has a designated contact person. Find the policy that you think is unfair of disadvantaging you, determine its contact person, and contact them regarding the policy. Again, you don't want to act confrontational: instead, state your case calmly and objectively.

If the policy's contact person isn't interested in changing policy, I'm afraid there's nothing more you can do. Accept defeat with good grace and adapt the club's activities to obey the appropriate policy.